\documentclass[10pt,a4paper]{article}
\usepackage{lastpage}
\usepackage{fancyhdr}
\pagestyle{fancy}
\usepackage{ dsfont }
\usepackage[a4paper, margin=0.3in]{geometry}
\usepackage{listings}
\usepackage{color}
\usepackage{xcolor}
\usepackage{adjustbox}
\usepackage{enumitem}

\setlist[enumerate]{itemsep=2pt, parsep=0pt}
\setlist[itemize]{itemsep=2pt, parsep=0pt}

\lstset{
    language=C++,                  % Set your language (you can change it as needed)
    basicstyle=\ttfamily\small,     % The basic style of your code
    keywordstyle=\color{blue},      % Keyword style
    stringstyle=\color{red},        % String literal style
    commentstyle=\color{green},     % Comment style
    morecomment=[l][\color{magenta}]{\#},  % Directive style (like #include)
    numbers=left,                   % Line numbers on left
    numberstyle=\tiny\color{gray},  % Line number style
    stepnumber=1,                   % Line number step
    numbersep=5pt,                  % How far the line numbers are from the code
    backgroundcolor=\color{white},  % Background color for code block
    tabsize=2,                      % Tab size
    showspaces=false,               % Show spaces adding particular underscores
    showstringspaces=false,         % Underline spaces within strings only
    showtabs=false,                 % Show tabs within strings adding particular underscores
    frame=single,                   % Adds a frame around the code
    captionpos=b,                   % Sets the caption-position to bottom
    breaklines=true,                % Sets automatic line breaking
    breakatwhitespace=false,        % Sets if automatic breaks should only happen at whitespace
    escapeinside={\%*}{*)},         % If you want to add LaTeX within your code
    morekeywords={*,...},           % If you want to add more keywords to the set
}
\newcommand{\code}[1]{\lstinline!#1!}

\begin{document}
\section*{Parser Design}
My parser design is quite unique amongst others, and results in clean, concise, bug-free and resilient code. 
\subsubsection*{Generating First and Follow sets programmatically}
I parse the grammar from a string form (\code{grammarString.hpp}), create a data structure to hold the productions for each nonterminal, and then compute the first and follow sets. The sets are then used to make the parser predictive - I converted the grammar to only require a lookahead of 1 (in almost every production), and so could use the sets to check the next production to take or whether to take an epsilon production.\\\\
Doing this programmatically avoided the inevitable errors of manual computation and ensures correctness and robustness of the compiler.
\subsubsection*{The \code{nonterminal} function}
I abstracted out the basic framework for a production into a function \code{nonTerminalInfo nonterminal(const std::string& name)}, which takes the name of the production to parse, and returns a map of strings to vectors of \code{ASTNode}s, for each nonterminal (and terminal) on the production's RHS. A \code{functionMap} is maintained so that the corresponding production function for a given nonterminal can be called dynamically. This removes the common and confusing spaghetti code involving \code{getNextToken()} and \code{putBackToken()} that would otherwise be present, resulting in very short, readable production functions. In fact \code{putBackToken()} was never even needed (and has been removed). As an example, here is the function for the \code{param} production:
\begin{lstlisting}[]
ptrVec<ASTNode> param() {
    nonTerminalInfo info = nonterminal("param");
    ptrVec<ASTNode> temp;
    ptrVec<ASTNode> var_type = std::move(info["var_type"]);
    ptrVec<ASTNode> ident = std::move(info["IDENT"]);
    temp.push_back(std::make_unique<ParamAST>(var_type, ident));
    return temp;
}
\end{lstlisting}
Everything is moved around as a general \code{ptrVec<ASTNode> = std::vector<std::unique_ptr<ASTNode>>} to facilitate this approach, until moved into an \code{ASTNode}s constructor, where nodes are cast and extracted as needed. Multiple overloaded templates, \code{castToDerived<Base, Derived>} were created to facilitate clean and easy casting of nodes.
\subsubsection*{ASTNodes}
\begin{table}[htbp]
    \centering
    \begin{adjustbox}{center}
    \begin{tabular}{|l|l|l|l|l|l|l|l|l|l|l|l|l|l|l|}
    \hline
    IntAST & FloatAST & BoolAST & FuncCallAST & VarCallAST & ProgramAST & NegationAST & StmtAST\\
    \hline
    DeclAST & BinOpAST & VarDeclAST & ParamAST & ExprStmtAST&FactorAST & StorageAST & PartialFuncDeclAST\\
    \hline
    IfAST & WhileAST & ExternAST & VarAssignAST & ReturnAST & BlockAST & FuncDeclAST & ExprAST\\
    \hline
    \end{tabular}
\end{adjustbox}
\end{table}
\noindent Generally, I found mirroring the structure of the grammar in my \code{ASTNode}s to be clearest. As a result, and in the service of polymorphism, there are some \code{ASTNode}s like \code{StmtAST}, \code{DeclAST}, and \code{ExprAST} for other nodes to derive from.
\code{StorageAST} was used to store terminal values in my implementation, including things like the value of \code{IDENT}.
\subsubsection*{Error handling}
Yet another benefit of my abstracted \code{nonterminal} function was that it allowed for a sublime way of implementing comprehensive error handling for the parser. I build up a \code{expected} set of terminals (using the first sets) as I check for productions to use - if no production can be found matching the current token, I pass this expected set to \code{throwParserError(const std::unordered_set<std::string>& expected, std::string found)}, which prints a comprehensive error:
\begin{lstlisting}[]
Encountered an error on line 2 column 23.
Expected one of: void int bool float extern 
Found: eextern
\end{lstlisting}
and then exits the program.\\
Similarly,
\code{throwCodegenError} makes use of tokens stored with each \code{ASTNode} to print the exact line, column and lexeme where an error occurred.
\section*{IR Generation}
Use of Finnbar Keating's lecture slides and the Kaleidescope tutorial were used to help find the right Build instructions and methods for some specific generations like function declarations. Beyond this, using BasicBlocks, setting insert points, and creating relevant instructions was a fairly intuitive process.
\subsubsection*{Casting and Type Checking}
Following the spirit of delightful code, multiple helper functions were created to make this area clean and robust. 
\begin{itemize}
    \item \code{stringToPtrType} and \code{ptrToStringType} to convert between \code{std::string} and \code{Type*} representations of types
    \item \code{Value* widenLtoR(Value* leftVal, Type* rightType)} which widens \code{leftVal} to \code{rightType} and returns the new corresponding \code{Value*}
    \item Dually, \code{Value* narrowLtoR(Value* leftVal, Type* rightType)}
\end{itemize}
Widening and narrowing were implemented using IRBuilder instructions \code{CreateSIToFP}, \code{CreateZExt}, \code{CreateFPToSI}, and \code{CreateICmpNE}. Following the coursework guide, as well as the C99 specification, the following are the rules that I created for type checking and casting:
\begin{itemize}
    \item Types are widened as much as needed during arithmetic and logical operations
    \item Types are only to be widened when passed into functions
    \item Types are only to be widened when returned from functions
    \item Types are widened or narrowed during variable assignment
\end{itemize}
Given the inverse of a rule, a descriptive error is thrown if needed.
\subsubsection*{Scope}
Using a vector of symbol tables for scope led to a fairly intuitive implementation - a new symbol table is pushed every time a new block is being code generated. A few tricks were required to completely make this approach work - for example, a global flag \code{isFunctionBlock}, so that we don't double push a symbol table when declaring a new function body.
\subsubsection*{Ensuring valid IR}
This involved ensuring:
\begin{enumerate}
    \item each basic block is terminated (ends in either a \code{ret} or \code{br} statement). Almost every block will naturally terminate. The only block which may not is the last block in our function body. If no return statement is present in this block, \code{verifyFunction()} returns true, and we either:
    \begin{enumerate}
        \item Add a void return for a void function
        \item Return a default value for a typed function
    \end{enumerate}
    This also results in a more intuitive experience for the end user.

    \item that no other instructions follow a \code{ret}. This is handled by stopping codegen of a block's statement list when a node of type \code{ReturnAST} is encountered. For cases where there may be a return statement within an if/else/while block, \code{hasCreatedReturn} is used to check if a branch should avoid being added to these blocks.
    \item that whenever loading from an alloca, there is a value stored in it. To solve this, I simply initialise all variables with a default value at first. This seemed a better, more intuitive solution to me than just allowing undefined behaviour as in the C99 standard.
\end{enumerate}
\subsubsection*{Lazy Boolean Evaluation}
My implementation of lazy evaluation is recursive and intuitive, leading to a robust and clean solution which is inductively correct. Here are the steps taken to implement lazy evaluation for the \code{&&} operator. \code{||} implementation is similar and so is omitted.
\begin{enumerate}
    \item Code generate the left node of the expression, and narrow the Value* returned to a boolean \code{Int1}.
    \item Create 3 Basic Blocks, leftFalseBB, evalRightBB, mergeBB.
    \item If the left value was false, we branch to leftFalseBB, otherwise we branch to evalRightBB.
    \item In leftFalseBB we simply branch to mergeBB immediately.
    \item in evalRightBB, we evaluate the right node, narrow the value, and then branch to mergeBB (remembering to reset the insert point which may have moved during right node evaluation).
    \item In mergeBB, we create a Phi node - if we come from predecessor leftFalseBB, then phi takes on the value \code{false}, otherwise coming from evalRightBB it takes the value of the right node.
    \item We simply return the phi value.
\end{enumerate}
So, as required, we only evaluate the right node if the left node returns true. Otherwise, we can short circuit the evaluation, and just return false.

\newpage

\bibliographystyle{IEEEtran}
\bibliography{report}

\newpage
\section*{Final Grammar}
{
\small
\begin{verbatim}
program ::= extern_list decl_list
program ::= decl_list
extern_list ::= extern extern_list'
extern_list' ::= extern extern_list'
extern_list' ::= ''
extern ::= "extern" type_spec IDENT "(" params ")" ";"
decl_list ::= decl decl_list'
decl_list' ::= decl decl_list'
decl_list' ::= ''
decl ::= var_type IDENT decl'
decl ::= "void" IDENT "(" params ")" block
decl' ::= "(" params ")" block
decl' ::= ";"
type_spec ::= "void"
type_spec ::= var_type
var_type  ::= "int"
var_type ::= "float"
var_type ::= "bool"
params ::= param_list
params ::= "void"
params ::= ''
param_list ::= param param_list'
param_list' ::= "," param param_list'
param_list' ::= ''
param ::= var_type IDENT
block ::= "{" local_decls stmt_list "}"
local_decls ::= local_decl local_decls
local_decls ::= ''
local_decl ::= var_type IDENT ";"
stmt_list ::= stmt stmt_list'
stmt_list ::= ''
stmt_list' ::= stmt stmt_list'
stmt_list' ::= ''
stmt ::= expr_stmt
stmt ::= block
stmt ::= if_stmt
stmt ::= while_stmt
stmt ::= return_stmt
expr_stmt ::= expr ";"
expr_stmt ::= ";"
while_stmt ::= "while" "(" expr ")" stmt
if_stmt ::= "if" "(" expr ")" block else_stmt
else_stmt ::= "else" block
else_stmt ::= ''
return_stmt ::= "return" return_stmt'
return_stmt' ::= ";"
return_stmt' ::= expr ";"
expr ::= IDENT "=" expr
expr ::= rval_or
rval_or ::= and_exp rval_or'
rval_or' ::= "||" and_exp rval_or'
rval_or' ::= ''
and_exp ::= equality_exp and_exp'
and_exp' ::= "&&" equality_exp and_exp'
and_exp' ::= ''
equality_exp ::= relational_exp equality_exp'
equality_exp' ::= "==" relational_exp equality_exp'
equality_exp' ::= "!=" relational_exp equality_exp'
equality_exp' ::= ''
relational_exp ::= additive_exp relational_exp'
relational_exp' ::= "<=" additive_exp relational_exp'
relational_exp' ::= "<" additive_exp relational_exp'
relational_exp' ::= ">=" additive_exp relational_exp'
relational_exp' ::= ">" additive_exp relational_exp'
relational_exp' ::= ''
additive_exp ::= multiplicative_exp additive_exp'
additive_exp' ::= "+" multiplicative_exp additive_exp'
additive_exp' ::= "-" multiplicative_exp additive_exp'
additive_exp' ::= ''
multiplicative_exp ::= factor multiplicative_exp'
multiplicative_exp' ::= "*" factor multiplicative_exp'
multiplicative_exp' ::= "/" factor multiplicative_exp'
multiplicative_exp' ::= "%" factor multiplicative_exp'
multiplicative_exp' ::= ''
factor ::= "-" factor
factor ::= "!" factor
factor ::= primary
primary ::= "(" expr ")"
primary ::= IDENT primary'
primary ::= INT_LIT
primary ::= FLOAT_LIT
primary ::= BOOL_LIT
primary' ::= "(" args ")"
primary' ::= ''
args ::= arg_list
args ::= ''
arg_list ::= expr arg_list'
arg_list' ::= "," expr arg_list'
arg_list' ::= ''
\end{verbatim}
}

\section*{First Sets}
{
\small
\begin{verbatim}
FIRST(stmt_list') = { '(' 'while' ';' '{' 'if' '!' 'FLOAT_LIT' 'BOOL_LIT' 'IDENT' '''' '-' 'return' 'INT_LIT' }
FIRST(relational_exp) = { '-' 'IDENT' 'INT_LIT' 'BOOL_LIT' '(' 'FLOAT_LIT' '!' }
FIRST(type_spec) = { 'float' 'void' 'int' 'bool' }
FIRST("else") = { 'else' }
FIRST(primary') = { '(' '''' }
FIRST("-") = { '-' }
FIRST(decl_list) = { 'bool' 'int' 'float' 'void' }
FIRST("%") = { '%' }
FIRST(multiplicative_exp') = { '/' '%' '''' '*' }
FIRST(relational_exp') = { '''' '<=' '<' '>=' '>' }
FIRST(stmt_list) = { ';' 'while' '{' '(' '!' 'if' 'FLOAT_LIT' 'BOOL_LIT' '''' 'return' '-' 'IDENT' 'INT_LIT' }
FIRST("!") = { '!' }
FIRST(else_stmt) = { '''' 'else' }
FIRST(INT_LIT) = { 'INT_LIT' }
FIRST(BOOL_LIT) = { 'BOOL_LIT' }
FIRST(factor) = { '(' 'FLOAT_LIT' '!' 'INT_LIT' 'IDENT' '-' 'BOOL_LIT' }
FIRST("<") = { '<' }
FIRST('') = { '''' }
FIRST(FLOAT_LIT) = { 'FLOAT_LIT' }
FIRST("/") = { '/' }
FIRST(arg_list') = { ',' '''' }
FIRST(multiplicative_exp) = { '-' 'IDENT' 'INT_LIT' 'BOOL_LIT' '(' 'FLOAT_LIT' '!' }
FIRST(return_stmt') = { 'INT_LIT' '-' 'IDENT' ';' 'BOOL_LIT' '!' 'FLOAT_LIT' '(' }
FIRST(extern_list') = { 'extern' '''' }
FIRST(additive_exp) = { '(' 'FLOAT_LIT' '!' 'INT_LIT' 'IDENT' '-' 'BOOL_LIT' }
FIRST(return_stmt) = { 'return' }
FIRST(equality_exp) = { '(' '!' 'FLOAT_LIT' 'INT_LIT' 'IDENT' '-' 'BOOL_LIT' }
FIRST("(") = { '(' }
FIRST(and_exp) = { '-' 'INT_LIT' 'IDENT' 'BOOL_LIT' '(' 'FLOAT_LIT' '!' }
FIRST(extern_list) = { 'extern' }
FIRST(rval_or') = { '||' '''' }
FIRST("*") = { '*' }
FIRST(";") = { ';' }
FIRST("+") = { '+' }
FIRST(expr) = { '-' 'IDENT' 'INT_LIT' 'BOOL_LIT' '(' 'FLOAT_LIT' '!' }
FIRST("while") = { 'while' }
FIRST(primary) = { 'FLOAT_LIT' '(' 'IDENT' 'BOOL_LIT' 'INT_LIT' }
FIRST(arg_list) = { '(' '!' 'FLOAT_LIT' 'INT_LIT' 'IDENT' '-' 'BOOL_LIT' }
FIRST(">=") = { '>=' }
FIRST("void") = { 'void' }
FIRST("{") = { '{' }
FIRST("int") = { 'int' }
FIRST(local_decl) = { 'float' 'bool' 'int' }
FIRST(while_stmt) = { 'while' }
FIRST(block) = { '{' }
FIRST(rval_or) = { 'IDENT' 'INT_LIT' '-' 'FLOAT_LIT' '!' '(' 'BOOL_LIT' }
FIRST(">") = { '>' }
FIRST(expr_stmt) = { '(' 'FLOAT_LIT' '!' 'INT_LIT' 'IDENT' '-' ';' 'BOOL_LIT' }
FIRST("if") = { 'if' }
FIRST(args) = { '-' 'IDENT' 'INT_LIT' '''' 'BOOL_LIT' '!' '(' 'FLOAT_LIT' }
FIRST(params) = { 'void' 'float' 'int' '''' 'bool' }
FIRST(local_decls) = { 'int' '''' 'bool' 'float' }
FIRST("==") = { '==' }
FIRST(if_stmt) = { 'if' }
FIRST(decl') = { ';' '(' }
FIRST("bool") = { 'bool' }
FIRST("extern") = { 'extern' }
FIRST(param) = { 'float' 'int' 'bool' }
FIRST("float") = { 'float' }
FIRST(IDENT) = { 'IDENT' }
FIRST(var_type) = { 'bool' 'int' 'float' }
FIRST(",") = { ',' }
FIRST("||") = { '||' }
FIRST(program) = { 'bool' 'int' 'void' 'float' 'extern' }
FIRST(equality_exp') = { '!=' '==' '''' }
FIRST("!=") = { '!=' }
FIRST(decl_list') = { 'int' '''' 'bool' 'void' 'float' }
FIRST(and_exp') = { '''' '&&' }
FIRST(decl) = { 'void' 'float' 'int' 'bool' }
FIRST(stmt) = { 'BOOL_LIT' '!' 'if' 'FLOAT_LIT' ';' 'IDENT' 'INT_LIT' 'return' '-' '(' 'while' '{' }
FIRST("<=") = { '<=' }
FIRST("return") = { 'return' }
FIRST(param_list') = { ',' '''' }
FIRST("&&") = { '&&' }
FIRST(extern) = { 'extern' }
FIRST(param_list) = { 'int' 'bool' 'float' }
FIRST(additive_exp') = { '+' '''' '-' }
\end{verbatim}
}

\section*{Follow Sets}
{
\small
\begin{verbatim}
FOLLOW(relational_exp) = { ';' ',' '||' '==' '&&' '!=' ')' }
FOLLOW(args) = { ')' }
FOLLOW(expr_stmt) = { 'BOOL_LIT' 'INT_LIT' 'IDENT' '-' 'return' 'FLOAT_LIT' '}' '{' 'if' '!' '(' 'while' ';' }
FOLLOW(else_stmt) = { 'BOOL_LIT' 'INT_LIT' 'IDENT' '-' 'return' 'FLOAT_LIT' '}' '{' 'if' '!' '(' 'while' ';' }
FOLLOW(arg_list') = { ')' }
FOLLOW(decl') = { 'bool' '0' 'float' 'void' 'int' }
FOLLOW(rval_or) = { ')' ';' ',' }
FOLLOW(equality_exp) = { ')' ';' ',' '||' '&&' }
FOLLOW(rval_or') = { ')' ';' ',' }
FOLLOW(and_exp) = { ')' ';' ',' '||' }
FOLLOW(multiplicative_exp) = { ',' '!=' '==' '>=' '>' '&&' '+' '<=' '||' '-' ';' ')' '<' }
FOLLOW(return_stmt') = { 'BOOL_LIT' 'INT_LIT' 'IDENT' '-' 'return' 'FLOAT_LIT' '}' '{' 'if' '!' '(' 'while' ';' }
FOLLOW(equality_exp') = { ')' ';' ',' '||' '&&' }
FOLLOW(program) = { '0' }
FOLLOW(stmt_list') = { '}' }
FOLLOW(relational_exp') = { ';' ',' '||' '&&' '==' '!=' ')' }
FOLLOW(additive_exp') = { ';' ',' '||' '!=' '<=' '>' '==' '&&' ')' '>=' '<' }
FOLLOW(arg_list) = { ')' }
FOLLOW(type_spec) = { 'IDENT' }
FOLLOW(multiplicative_exp') = { ',' '!=' '==' '>=' '>' '&&' '<=' '+' ')' '||' '-' ';' '<' }
FOLLOW(stmt) = { 'BOOL_LIT' 'INT_LIT' 'IDENT' '-' 'return' 'FLOAT_LIT' '}' '{' '!' 'if' ';' '(' 'while' }
FOLLOW(stmt_list) = { '}' }
FOLLOW(expr) = { ')' ';' ',' }
FOLLOW(return_stmt) = { 'BOOL_LIT' 'INT_LIT' 'IDENT' '-' 'return' 'FLOAT_LIT' '}' '{' 'if' '!' '(' 'while' ';' }
FOLLOW(additive_exp) = { ';' ',' '||' '>=' '<=' '!=' '>' '==' '&&' '<' ')' }
FOLLOW(local_decls) = { 'BOOL_LIT' 'INT_LIT' 'return' '-' 'IDENT' 'FLOAT_LIT' '}' 'while' '{' '(' '!' 'if' ';' }
FOLLOW(block) = { 'INT_LIT' '-' 'return' 'IDENT' 'BOOL_LIT' 'float' 'int' 'else' 'FLOAT_LIT' '}' 'bool' '{' 'while' 'void' 'if' '!' '(' ';' }
FOLLOW(params) = { ')' }
FOLLOW(decl) = { 'bool' 'float' 'void' 'int' }
FOLLOW(primary) = { ',' '&&' '>' '>=' '==' '<=' '+' ')' '||' '-' '<' '%' '*' '/' ';' '!=' }
FOLLOW(extern_list) = { 'void' 'float' 'bool' 'int' }
FOLLOW(decl_list) = { '0' }
FOLLOW(if_stmt) = { 'BOOL_LIT' 'INT_LIT' 'IDENT' '-' 'return' 'FLOAT_LIT' '}' '{' 'if' '!' '(' 'while' ';' }
FOLLOW(decl_list') = { '0' }
FOLLOW(and_exp') = { ')' ';' ',' '||' }
FOLLOW(factor) = { ',' '&&' '>' '>=' '==' '<=' '+' ')' '||' '-' '<' ';' '/' '!=' '%' '*' }
FOLLOW(param_list) = { ')' }
FOLLOW(param_list') = { ')' }
FOLLOW(primary') = { ',' '&&' '>' '>=' '==' '<=' '+' ')' '||' '-' '<' '%' '*' '/' ';' '!=' }
FOLLOW(param) = { ')' ',' }
FOLLOW(extern_list') = { 'void' 'float' 'bool' 'int' }
FOLLOW(var_type) = { 'IDENT' }
FOLLOW(local_decl) = { 'INT_LIT' 'IDENT' 'return' '-' 'int' '}' 'FLOAT_LIT' 'bool' '!' 'if' '(' ';' 'float' 'BOOL_LIT' 
'while' '{' }
FOLLOW(while_stmt) = { 'BOOL_LIT' 'INT_LIT' 'IDENT' '-' 'return' 'FLOAT_LIT' '}' '{' 'if' '!' '(' 'while' ';' }
FOLLOW(extern) = { 'void' 'int' 'bool' 'float' 'extern' }    
\end{verbatim}
}

\end{document}